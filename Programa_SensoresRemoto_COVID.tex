%#############################PREAMBLE#############################################
\documentclass[a4paper,twoside,11pt,]{article}

\usepackage[spanish]{babel}
\usepackage{graphicx}
\usepackage{float}
\usepackage[skins]{tcolorbox}
\usepackage{titlepic}
\renewcommand{\labelenumi}{Clase \theenumi}

\usepackage{fancyhdr}
\pagestyle{fancy}
\lhead{Sensores Remotos}
\rhead{\thepage}
\cfoot{Programa}
\renewcommand{\headrulewidth}{0.4pt}
\renewcommand{\footrulewidth}{0.4pt}

\graphicspath{{G:/My Drive/FIGURAS/}}

\title {PROGRAMA  CURSO SENSORES REMOTOS}
\author{Prof.: Edier Aristizábal\\[5ex]
\includegraphics[width=5.0cm]{un_verde}
}
\date{}

%################################BODY############################################
\begin{document}
\maketitle

\emph {versión}: \today

\section*{NOTA:}
Debido a la situación especial en la cual se dictará este curso, como consecuencia de la pandemia del COVID-19, el contenido regular del curso ha sido modificado,
de tal forma que se reduzcan los talleres con el uso del estereoscopio. En este orden de ideas, el curso se enfocará en el uso de herramientas
digitales, y se conservan dos talleres presenciales al final del curso con el uso del estereoscopio de espejos.

\section* {Introducción}
El curso de \emph{Sensores Remotos} está orientado para estudiantes de ciencias de la tierra con el objeto de aprender a utilizar las herramientas de teledetección en geología y geomorfología. Inicialmente comprende la teoría general de sensores remotos y procesamiento de imágenes. Para posteriormente enfocarse en el uso de fotografías aéreas y adquirir de forma adecuada la técnica de la fotointerpretación a través del uso de fotografías aéreas.\\ 
Este curso no corresponde a un curso a profundidad y detalle del uso de imágenes de satélite para diferentes disciplinas. De forma similar, la técnica de fotointerpretación, aunque es similar para otros temas, su aplicación en este curso se enfoca exclusivamente para la fotointerpretación geológica, es decir diferenciar unidades litológicas, al igual que fotointerpretación geomorfológica, es decir formas y procesos morfodinámicos.\\
El procesamiento de imágenes es una herramienta ampliamente utilizada actualmente, y la fotointerpretación es una técnica que se conserva por su ayuda en diferentes campos, y que no puede suplir ningún otro sensor remoto. Adquirir estas herramientas seguramente le ampliará sus perspectivas profesionales en el campo de la geología aplicada a la ingeniería.

\section{RECOMENDACIÓN}
Para tomar el curso se recomienda al estudiante haber realizado su núcleo básico y los cursos SIG, Campo I, Geomorfología, Rocas Sedimentarias, Rocas Metamórficas, Rocas Ígneas, y Geología Estructural. De esta forma el estudiante podrá sacar el máximo beneficio del contenido del curso.\\
En caso que el estudiante no haya cursado las anteriores asignaturas se recomienda que cancele el curso. Es posible que tome el curso y lo apruebe, con mucha mayor dificultad, sin embargo no podrá explotar todas las posibilidades que ofrecen los sensores remotos.

\section{PROGRAMA}
El contenido del curso comprende los siguientes temas a desarrollar:\\

\subsection*{Introducción al curso}

\subsection {Radiación electromagnética}
\begin{itemize}
\item Energía electromagnética y espectro
\item Interacciones con la atmosfera
\item Absorción y transmisión
\item Dispersión
\item Interacción con la superficie
\item Cálculo de la reflectividad
\item Reflectancia vs. Radiancia
\item Curvas de reflectividad
\item Características de las imágenes
\end{itemize}

\begin{tcolorbox}[enhanced,width=5in,center upper,  fontupper=\large\bfseries,drop shadow southwest,sharp corners]
Taller 1 -- Descarga de imágenes
\end{tcolorbox}

\subsection {Radiación electromagnética}
\begin{itemize}
\item Interacción con el objeto
\item Reflexión
\item Emisión
\item Dispersión
\item Transmisión
\item Curvas de reflectancia espectral
\item Superficie Especular y Lambertiana
\item Firma espectral (agua, suelo, vegetación)
\end{itemize}

\begin{tcolorbox}[enhanced,width=5in,center upper,  fontupper=\large\bfseries,drop shadow southwest,sharp corners]
Taller 2 -- Manejo imágenes satelitales
\end{tcolorbox}

\subsection {Sensores: plataformas y detectores}
\begin{itemize}
\item Plataformas aéreas y espaciales (Airborne vs spaceborne)
\item Orbitas
\item Tipo de sensores
\item Sensores Pasivos
\item Explorador de barrido
\item Explorador de empuje
\end{itemize}

\begin{tcolorbox}[enhanced,width=5in,center upper,  fontupper=\large\bfseries,drop shadow southwest,sharp corners]
Taller 3 -- Indices espectrales
\end{tcolorbox}

\subsection {Sensores: resolución}
\begin{itemize}
\item Las resoluciones
\item IFOV
\item Resolución radiométrica
\item Resolución espectral
\item Resolución geométrica
\item Resolución temporal
\item Resolución vs Escala
\item Sensores Activos
\end{itemize}

\begin{tcolorbox}[enhanced,width=5in,center upper,  fontupper=\large\bfseries,drop shadow southwest,sharp corners]
Taller 4 -- Presentación programas espaciales
\end{tcolorbox}

\subsection {Tratamiento de imágenes}
\begin{itemize}
\item Numero digital
\item Tamaño de la imagen
\item Formato de grabación
\item Error y calibración radiométrico
\item Error y calibración geométrica
\item Procesamiento de imágenes
\item Transformación de imágenes
\item Clasificación de imágenes
\end{itemize}

\begin{tcolorbox}[enhanced,width=5in,center upper,  fontupper=\large\bfseries,drop shadow southwest,sharp corners]
Taller 5 -- Clasificación de imágenes
\end{tcolorbox}

\subsection {Evaluación}
\begin{itemize}
\item Matriz de confusión
\item Precisión
\item Precisión del usuario
\item Precisión del productor
\item Coeficiente de Kappa Cohen
\end{itemize}

\begin{tcolorbox}[enhanced,width=5in,center upper,  fontupper=\large\bfseries,drop shadow southwest,sharp corners]
Taller 6 --Evaluación de la clasificación de imágenes satelitales
\end{tcolorbox}

\subsection {Introducción a la fotografía aérea}
\begin{itemize}
\item Historia
\item Tipos de fotointerpretación
\item Estereoscopio
\item Visión estereoscópica
\item El efecto GESTAL y la percepción
\end{itemize}

\begin{tcolorbox}[enhanced,width=5in,center upper,  fontupper=\large\bfseries,drop shadow southwest,sharp corners]
Taller 7 -- Intro al espereoscopio
\end{tcolorbox}

\subsection {Criterios de Fotointerpretación}
\begin{itemize}
\item Técnicas y métodos de fotointerpretación 
\item Principios básicos: tamaño, forma, tono o color, textura, patrón, sombra y asociación
\item Elementos básicos (laderas), compuestos (drenaje, fallas y lineamientos), e inferidos (erosión, roca parental)
\end{itemize}

\begin{tcolorbox}[enhanced,width=5in,center upper,  fontupper=\large\bfseries,drop shadow southwest,sharp corners]
Taller 8 -- Fotointerpretación asistida
\end{tcolorbox}

\subsection {\emph{Google Earth Engine Explorer}}
\begin{itemize}
\item Bases de datos
\item Análisis multitemporales
\item Descarga
\item Cálculo de índices
\item Clasificación de imágenes de satelite
\end{itemize}

\begin{tcolorbox}[enhanced,width=5in,center upper,  fontupper=\large\bfseries,drop shadow southwest,sharp corners]
Taller 9 -- GEE
\end{tcolorbox}

\subsection {Análisis geoespacial con Python}
\begin{itemize}
\item Ambiente computacional
\item Librerias
\item Geodataframe
\item Análisi espacial
\end{itemize}

\begin{tcolorbox}[enhanced,width=5in,center upper,  fontupper=\large\bfseries,drop shadow southwest,sharp corners]
Taller 10 -- Python
\end{tcolorbox}

\begin{tcolorbox}[enhanced,width=5in,center upper,  fontupper=\large\bfseries,drop shadow southwest,sharp corners]
Taller 11 -- Cartografía
\end{tcolorbox}

\section{TALLERES}
La descripción de cada uno de los talleres se encuentra en el Moodle. Las fechas de entrega y porcentaje se presentan a continuación (La hora de entrega de todos los talleres es hasta las 8:00 a.m.):\\
\begin{table}[!hbt]
\label{tab-marks}
\begin{tabular}{|l|c|c|c|}
\hline {\bf Talleres} & {\bf Fecha entrega} & {\bf Evaluación (\%)} & {\bf Tipo} \\
\hline Taller 1 Descarga de imágenes & 16/03/2021 &  5 & Individual\\
\hline Taller 2 Imágenes en ArcGIS &  23/03/2021& 5 & Individual\\
\hline Taller 3 NVDI & 30/03/2021  & 5 & Individual\\
\hline Taller 4 Programas espaciales  & 06/04/2021 &  5 & Grupo de 4\\
\hline Taller 5 Clasificación &  13/04/2021 & 10 & Individual\\
\hline Taller 6 Evaluación & 20/04/2021  & 5 & Individual\\
\hline Taller 7 (los de campo) Intro al estereoscopio & 05/09/2021 &  10 & Individual\\
\hline Taller 7 (los no campo) Fotointerpretación con anaglifo & 05/09/2021 &  10 & Individual\\
\hline Taller 8 GEE &  05/09/2021& 10 & Individual\\
\hline Taller 9 Python  & 05/09/2021  & 10 & Individual\\
\hline Taller 10. Cartografia geologica & 05/09/2021  & 10 & Individual\\
\hline Taller 11. Trabajo final &05/09/2021 & 25 & Grupo de 4\\
\hline
\end{tabular}
\end{table}


Los talleres deberán cargarse a la plataforma de Google Classroom en formato PDF. El nombre del archivo deberá tener el número del taller y el nombre y apellido del estudiante (Ej. Taller 1\_EdierAristizabal). En caso de no entregarse de esta forma tendrá un descuento del 0,5 de la nota obtenida.
El Taller 1 comprende la descarga de una imagen de satélite Landsat-8 tomada durante el semestre actual. Con dicha imagen se deberá continuar realizando los talleres con imágenes de satélite siguientes. El uso de una imagen diferente para los siguientes talleres implica que sean evaluados sobre una nota máxima de 3.0.
La presentación de los talleres tiene un formato libre, el cual exige un trabajo de creatividad, orden y claridad del estudiante, de tal forma que trasmita la información de forma correcta y adecuada al evaluador. Los criterios utilizados para la evaluación y asignación de nota a los talleres son:
\begin{itemize}
\item Ejecución: que el taller se presente completo y desarrolle todas las actividades solicitadas.
\item Claridad y orden: que se desarrolle de forma secuencial y clara en términos de la estructura, redacción y figuras o tablas de apoyo.
\item Conciso: que pueda transmitir la información suficiente de forma corta y directa.
\item Desarrollo adecuado: que el procedimiento, análisis y observaciones realizadas sean correctas de acuerdo al contenido y estado del arte del curso.
\end{itemize}

\section{REFERENCIAS}
El curso utilizará material de diferentes fuentes bibliográficas, entre las cuales se destacan las siguientes, por lo cual se recomienda su consulta:
\begin{itemize}
\item Aerial photographs in geologic interpretation and mapping. US Geological Survey professional paper 373. Richard Ray (Eds). 1960.
\item Fundamentals of remote sensing. A Canada centre for remote sensing tutorial. Natural Resources Canada. 
\item Principles of remote sensing, an introductory textbook. The International Institute for Geo-Information Science and Earth Observation (ITC). 2009.
\item Guía Teórica de Fotogeología. Gutierrez A. Julian. Universidad de Los Andes, Merida, Venezuela
\item Manual de ejercicios de laboratorio fotogrametría y fotointerpretación. Carlos Pacheco \& Ennio Pozzobon. Universidad de Los Andes. 2006.
\end{itemize}

\end{document}
