%###########################PRESENTACION##########################################
%Modo presentación
\documentclass[]{beamer}

%Modo handout
%\documentclass[handout,compress]{beamer}
%\usepackage{pgfpages}
%\pgfpagesuselayout{4 on 1}[border shrink=1mm]

\usepackage{graphicx}
\usepackage{beamerthemeCambridgeUS}
\usepackage{subfig}
\usepackage{tikz}
\usepackage{amsmath}
\usepackage{ragged2e}
\setbeamercovered{transparent}

\graphicspath{{G:/My Drive/FIGURAS/}}

\title[Sensores]{SENSORES REMOTOS}
\author[Edier Aristizábal]{Edier V. Aristizábal G.}
\institute{\emph{evaristizabalg@unal.edu.co}}
\date{(Versión:\today)}
\usepackage{textpos} 

\addtobeamertemplate{headline}{}{%
	\begin{textblock*}{2mm}(.9\textwidth,0cm)
	\hfill\includegraphics[height=1cm]{un}  
	\end{textblock*}
			}
%############################INICIO#############################################
\begin{document}
%###########################SLIDE
\begin{frame}
\titlepage
\centering
	\includegraphics[width=5cm]{unal}\hspace*{4.75cm}~%
   	\includegraphics[width=2cm]{logo3} 
\end{frame}
%#############################SLIDE
\begin{frame}
\begin{center}
\includegraphics[scale=0.40]{programasespaciales}
\end{center}
\end{frame}
%#############################SLIDE
\begin{frame}
\frametitle{Orbitas polares}
\begin{center}
\includegraphics[scale=0.7]{orbitaspolares}
\end{center}
\end{frame}
%#############################SLIDE
\begin{frame}
\frametitle{Orbitas polares}
\begin{center}
\includegraphics[scale=0.55]{orbitaspolares1}
\end{center}
\end{frame}
%#############################SLIDE
\begin{frame}
\frametitle{Orbitas Geoestacionario}
\begin{center}
\includegraphics[scale=0.55]{orbitaspolares1}
\end{center}
\end{frame}
%#############################SLIDE
\begin{frame}
\frametitle{Orbitas Geoestacionarias vs. Heliosincrónicas}
\begin{center}
\includegraphics[scale=0.5]{orbitas}
\end{center}
\end{frame}
%#############################SLIDE
\begin{frame}
\frametitle{Plataformas}
\begin{center}
\includegraphics[scale=0.42]{plataformas}
\end{center}
\end{frame}
%################################SLIDE
\begin{frame}
\frametitle{Swath \& Path}
\begin{columns}
\begin{column}{0.5\linewidth}
\begin{center}
\includegraphics[scale=0.4]{swaths1}
\end{center}
\end{column}
\begin{column}{0.5\linewidth}
\begin{center}
\includegraphics[scale=0.65]{swaths}
\end{center}
\end{column}
\end{columns}
\end{frame}
%################################SLIDE
\begin{frame}
\frametitle{Landsat 8: Swath \& Path}
\begin{center}
\includegraphics[scale=0.42]{swaths8}
\end{center}
\end{frame}
%################################SLIDE
\begin{frame}
\frametitle{Satélite Bus: Landsat 8}
\begin{center}
\includegraphics[scale=0.40]{landsat8_bus}
\end{center}
\end{frame}
%################################SLIDE
\begin{frame}
\frametitle{Detectores Análogos vs Digitales}
\begin{center}
\includegraphics[scale=0.6]{analoga_digital1}
\end{center}
\end{frame}
%################################SLIDE
\begin{frame}
\frametitle{Detectores Análogos vs Digitales}
\begin{center}
\includegraphics[scale=0.40]{analoga_digital}
\end{center}
\end{frame}
%################################SLIDE
\begin{frame}
\frametitle{Detectores}
\small{Los detectores son definidos como instrumentos que reciben un flujo de energía y proporcionan una señal. Existen dos tipos fundamentales de detectores de luz que operan con mecanismos de transducción diferentes.
\begin{itemize}
\item Cuánticos (Photon) (CCD – CMOS) 
\item Térmicos 
\end{itemize}
}
\begin{center}
\includegraphics[scale=0.38]{detectores}
\end{center}
\end{frame}
%################################SLIDE
\begin{frame}
\frametitle{Detectores}
\framesubtitle{CCD (Charge-Coupled Device) - CMOS (Complementary Metal-Oxide Semicoductor)}
\begin{center}
\includegraphics[scale=0.6]{detectores1}
\end{center}
\end{frame}
%################################SLIDE
\begin{frame}
\frametitle{Detectores termales}
\begin{center}
\includegraphics[scale=0.32]{thermal}
\end{center}
\end{frame}
%################################SLIDE
\begin{frame}
\frametitle{Teoría del color}
\framesubtitle{Teoria aditiva (RGB) \& Teoría sustractiva (CMYK)}
\begin{center}
\includegraphics[scale=0.6]{teoriacolor}
\end{center}
\end{frame}
%################################SLIDE
\begin{frame}
\frametitle{Tipo de sensores}
\begin{center}
\includegraphics[scale=0.45]{sensores}
\end{center}
\end{frame}
%################################SLIDE
\begin{frame}
\frametitle{Espectrómetros}
\begin{center}
\includegraphics[scale=0.5]{espectrometros}
\end{center}
\end{frame}
%################################SLIDE
\begin{frame}
\frametitle{Elementos para dispersión}
\begin{center}
\includegraphics[scale=0.45]{elementos_dispersion}
\end{center}
\end{frame}
%################################SLIDE
\begin{frame}
\frametitle{Sensores Activos \& Pasivos}
\begin{center}
\includegraphics[scale=0.45]{sensores_activos_pasivos}
\end{center}
\end{frame}
%################################SLIDE
\begin{frame}
\frametitle{Sensores Activos \& Pasivos}
\begin{center}
\includegraphics[scale=0.45]{sensores_activos_pasivos1}
\end{center}
\end{frame}
%################################SLIDE
\begin{frame}
\frametitle{Tipo cámaras}
\begin{center}
\includegraphics[scale=0.45]{tipo_camaras}
\end{center}
\end{frame}
%################################SLIDE
\begin{frame}
\frametitle{Tipo Frame}
\begin{center}
\includegraphics[scale=0.45]{frame1}
\end{center}
\end{frame}
%################################SLIDE
\begin{frame}
\frametitle{Escaner}
\framesubtitle{Barrido vs Empuje}
\begin{center}
\includegraphics[scale=0.45]{escaner}
\end{center}
\end{frame}
%################################SLIDE
\begin{frame}
\frametitle{Empuje}
\begin{center}
\includegraphics[scale=0.42]{empuje}
\end{center}
\end{frame}
%################################SLIDE
\begin{frame}
\frametitle{Combinación}
\begin{center}
\includegraphics[scale=0.42]{tipo_camaras1}
\end{center}
\end{frame}
%################################SLIDE
\end{document}