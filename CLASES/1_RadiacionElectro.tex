%###########################PRESENTACION##########################################
%Modo presentación
\documentclass[]{beamer}

%Modo handout
%\documentclass[handout,compress]{beamer}
%\usepackage{pgfpages}
%\pgfpagesuselayout{4 on 1}[border shrink=1mm]

\usepackage{graphicx}
\usepackage{beamerthemeCambridgeUS}
\usepackage{subfig}
\usepackage{tikz}
\usepackage{amsmath}
\usepackage{ragged2e}
\setbeamercovered{transparent}

\graphicspath{{G:/My Drive/FIGURAS/}}

\title[Radiación Electromagnética]{SENSORES REMOTOS}
\author[Edier Aristizábal]{Edier V. Aristizábal G.}
\institute{\emph{evaristizabalg@unal.edu.co}}
\date{(Versión:\today)}
\usepackage{textpos} 

\addtobeamertemplate{headline}{}{%
	\begin{textblock*}{2mm}(.9\textwidth,0cm)
	\hfill\includegraphics[height=1cm]{un}  
	\end{textblock*}
			}
%############################INICIO#############################################
\begin{document}
%###########################SLIDE
\begin{frame}
\titlepage
\centering
	\includegraphics[width=5cm]{unal}\hspace*{4.75cm}~%
   	\includegraphics[width=2cm]{logo3} 
\end{frame}
%#############################SLIDE
\begin{frame}
\begin{center}
\includegraphics[scale=0.5]{sensoresremotos4}
\end{center}
\end{frame}
%#############################SLIDE
\begin{frame}
\frametitle{Transferencia de energía}
\justifying
\small{
Existen tres formas diferentes de transferencia energética:\\
\vfill
\textbf{Conducción}: es el mecanismo de transferencia de calor en escala atómica. Se produce por la vibración y  choque de unas moléculas con otras, donde las partículas más energéticas le entregan energía a las menos energéticas.\\
\vfill
\textbf{Convección}: mecanismo de transferencia de calor por movimiento de masa o circulación dentro de la sustancia, es propia de fluidos (líquidos o gaseosos) en movimiento.\\
\vfill 
\color{blue}
\textbf{Radiación}: es energía emitida por la materia que se encuentra a una temperatura dada. Esta energía es producida por los cambios en las configuraciones electrónicas de los átomos o moléculas. Esta energía es transportada por ondas electromagnéticas o fotones, por lo recibe el nombre de radiación electromagnética que se propaga a través del vacío y a la velocidad de la luz. 
}
\end{frame}
%#############################SLIDE
\begin{frame}
\frametitle{Fuente de Anergía}
\framesubtitle{Energía electromagnética $\rightarrow$ doble naturaleza = onda y partícula}
\begin{center}
\includegraphics[scale=0.5]{wave3}
\end{center}
\end{frame}
%#############################SLIDE
\begin{frame}
\frametitle{Fuente de Anergía}
\framesubtitle{Energía electromagnética $\rightarrow$ doble naturaleza = onda y partícula}
\begin{center}
Young's experiment
\includegraphics[scale=0.4]{interfero}
\end{center}
\end{frame}
%#############################SLIDE
\begin{frame}
\frametitle{Fuente de Anergía}
\framesubtitle{Energía electromagnética $\rightarrow$ doble naturaleza = onda y partícula}
\begin{center}
\includegraphics[scale=0.4]{planck}
\includegraphics[scale=0.5]{wave4}
\end{center}
\end{frame}
%#############################SLIDE
\begin{frame}
\frametitle{Espectro electromagnético}
\begin{center}
\includegraphics[scale=0.5]{wave2}
\end{center}
\end{frame}
%#############################SLIDE
\begin{frame}
\frametitle{Información obtenida}
\begin{center}
\includegraphics[scale=0.5]{wave_data}
\end{center}
\end{frame}
%#############################SLIDE
\begin{frame}
\frametitle{Balance de Energía}
\begin{center}
\includegraphics[scale=0.5]{balance_energia}
\end{center}
\end{frame}
%#############################SLIDE
\begin{frame}
\frametitle{Fuente de Energía}
\framesubtitle{Ley de Stefan-Boltzmann \& Ley de Wien}
  \begin{figure}
    \centering
    \includegraphics[height=.7\textheight]{stefan}
    %\caption{This is the caption.}
  \end{figure}
\end{frame}
%################################SLIDE
\begin{frame}
\frametitle{Espectro Electromagnético}
  \begin{figure}
    \centering
    \includegraphics[height=.7\textheight]{espectro}
    %\caption{This is the caption.}
  \end{figure}
\end{frame}
%################################SLIDE
\begin{frame}
\frametitle{Espectro Electromagnético}
  \begin{figure}
    \centering
    \includegraphics[height=.7\textheight]{espectro3}
    %\caption{This is the caption.}
  \end{figure}
\end{frame}
%################################SLIDE
\begin{frame}
\frametitle{Unidades}
\begin{itemize}
\scriptsize{
\item \textbf{Energía radiante}: total de energía radiada en todas las direcciones (J).
\item \textbf{Flujo radiante}: energía radiada en todas las direcciones por unidad de tiempo (W).
\item \textbf{Irradiancia}: flujo radiante incidente sobre unidad de área ($\frac{w}{m^2}$).
\item \textbf{Radiancia}: flujo radiante emitido o reflejado por unidad de área y por ángulo solido de medida($\frac{w*Sr}{m^2}$).
\item \textbf{Emisividad}: relación entre la emitancia y la de un emisor perfecto.
\item \textbf{Reflectividad}: relación entre el flujo incidente y el flujo reflejado por una superficie.
\item \textbf{Absortividad}: relación entre el flujo incidente y el flujo que absorbe una superficie.
\item \textbf{Trasmisividad}: relación entre el flujo incidente y el transmitido por una superficie.
}
 \begin{figure}
    \centering
    \includegraphics[height=.5\textheight]{radiance}
  \end{figure}
\end{itemize}
\end{frame}
%%%%%%%%%%%%%%%%%%%%%%%%%%%%%%%%%%%%%%%%%%%%%%%%%%%%%%%%%%%%
\begin{frame}
\frametitle{Interacción con la atmósfera}
%\framesubtitle{}
  \begin{figure}
    \centering
    \includegraphics[height=.7\textheight]{atmosfera}
  \end{figure}
\end{frame}
%################################SLIDE
\begin{frame}
\frametitle{Interacción con la atmósfera}
%\framesubtitle{}
  \begin{figure}
    \centering
    \includegraphics[height=.7\textheight]{atmosfera1}
  \end{figure}
\end{frame}
%################################SLIDE
\begin{frame}
\frametitle{Atenuación}
  \begin{columns}
		\begin{column}{.4\linewidth}
		 \scriptsize{\textbf{Atenuación geométrica}: Atenuación de la radiación con la distancia recorrida desde la fuente hasta el receptor. La intensidad de la onda decrece usualmente con el cuadrado de la distancia al foco emisor.\\
\vspace{5pt}
\textbf{Atenuación atmosférica}: Atenuación por absorción de las moléculas atmosféricas. La intensidad de la onda disminuye exponencialmente con la distancia r del medio absorbente atravesado.}
		\end{column}
		\begin{column}{.6\linewidth}
			 \includegraphics[height=.5\textheight]{atenuacion}
		\end{column}
	\end{columns}
\end{frame}
%################################SLIDE
\begin{frame}
\frametitle{Ventanas Atmosféricas}
\begin{center} 
\includegraphics[width=8cm]{ventanas}
\end{center}
\tiny{Credit: NASA's Imagine the Universe}
\end{frame}
%################################SLIDE
\begin{frame}
\frametitle{Dispersión}
  \begin{columns}
		\begin{column}{.3\linewidth}
		 \includegraphics[width=3cm]{ray}
		\end{column}
		\begin{column}{.7\linewidth}
\scriptsize{\textbf{Rayleigh}: Dispersión dominante en la atmósfera. Diámetro de las partículas es inferior a la longitud de onda. Longitudes de ondas menores (azul) es mas disperso que longitudes de onda mayores (rojo). El efecto Rayleigh es inversamente proporcional a la longitud de onda a la 4. Se utilizan filtros que eliminan longitudes de onda corta.\\
(moléculas atmosféricas: O3, N2, CO2, etc)}.
		\end{column}
	\end{columns}
	\begin{columns}
		\begin{column}{.3\linewidth}
		 \includegraphics[width=4cm]{mie}
		\end{column}
		\begin{column}{.7\linewidth}
\scriptsize{\textbf{Mie}: Longitudes de onda similares al diámetro de las partículas, característico en la parte baja de la atmósfera. Mayor efecto en longitudes de onda mas grandes comparado con efecto Rayleigh.\\
(Polvo, Vapor de agua).}
		\end{column}
	\end{columns}
	\begin{columns}
	 	\begin{column}{.3\linewidth}
		 \includegraphics[width=4cm]{G:/My Drive/CATEDRA/ANALISIS GEOESPACIAL/fig/noselectiva}
		\end{column}
		\begin{column}{.7\linewidth}
\scriptsize{\textbf{No selectiva}. Diámetro (5  a 100 um) de las partículas es mucho mayor a las longitudes de onda, por lo que dispersa todo el espectro del visible y hasta el infrarrojo medio.\\
(Aerosoles, gotas de lluvia).}
		\end{column}
	\end{columns}
\end{frame}
%################################SLIDE
\begin{frame}
\frametitle{Dispersión selectiva}
\begin{center} 
\includegraphics[scale=0.55]{bluesky}
\end{center}
\end{frame}
%################################SLIDE
\begin{frame}
\frametitle{Dispersión NO selectiva}
\begin{center} 
\includegraphics[scale=0.8]{cloud}
\end{center}
\end{frame}
%################################SLIDE
\begin{frame}
\frametitle{Interacción con el objeto}
  \begin{figure}
    \centering
    \includegraphics[height=.5\textheight]{objeto}
  \end{figure}
  \begin{columns}
	 	\begin{column}{.5\linewidth}
		 \includegraphics[width=5cm]{reflejo}
		\end{column}
		\begin{column}{.5\linewidth}
		\includegraphics[width=4cm]{ecuacion}
		\end{column}
	\end{columns}
\tiny{}
\end{frame}
%################################SLIDE
\begin{frame}
\scriptsize{Se pueden distinguir dos tipos de superficies:\vfill\textbf{Especulares}. Aquellas que reflejan la energía con el mismo ángulo del flujo incidente.\vfill
\textbf{Lambertianas--difusa}.  Aquellas que reflejan el flujo incidente uniformemente en todas las direcciones. En Sensores Remotos generalmente es de mayor interés medir las propiedades de reflectancia difusa del terreno y objetos.} 
  \begin{figure}
    \centering
    \includegraphics[height=.5\textheight]{objeto2}
  \end{figure}
\tiny{}
\end{frame}
%################################SLIDE
\begin{frame}
%\framesubtitle{}
  \begin{figure}
    \centering
    \includegraphics[height=.6\textheight]{dispersion2}
  \end{figure}
\end{frame}
%################################SLIDE
\begin{frame}
\frametitle{Relación angular entre Imagen - Objeto - Sol}
\begin{columns}
\begin{column}{0.5\linewidth}
  \begin{figure}
    \centering
    \includegraphics[height=.6\textheight]{solarangle}
  \end{figure}
  \end{column}
 \begin{column}{0.5\linewidth}
 \begin{figure}
    \centering
    \includegraphics[height=.8\textheight]{solarangle1}
  \end{figure} 
 \end{column}
  \end{columns}
\end{frame}
%################################SLIDE
\begin{frame}
\frametitle{Relación angular entre Imagen - Objeto - Sol}
  \begin{figure}
    \centering
    \includegraphics[scale=0.5]{solarangle2}
  \end{figure}
\end{frame}
%################################SLIDE
\begin{frame}
%\framesubtitle{}
  \begin{figure}
    \centering
    \includegraphics[height=.8\textheight]{especular}
  \end{figure}
\end{frame}
%################################SLIDE
\begin{frame}
\frametitle{Características de las coberturas}
\scriptsize{
La forma de reflejar la energía en las distintas longitudes de onda no es único y homogéneo, sino que varía sustancialmente en función de los siguientes factores:\vfill
\textbf{Físicos}: temperatura, humedad y textura.\vfill
\textbf{Químicos}: composición, contenido de materia orgánica, etc.\vfill
\textbf{Ambientales}: pendiente, orientación, estación del año, hora, etc.
}
 \begin{figure}
    \centering
    \includegraphics[height=.6\textheight]{paisaje}
  \end{figure}
\end{frame}
%%%%%%%%%%%%%%%%%%%%%%%%%%%%%%%%%%%%%%%%%%%%%%%%%%%%%%%%%%%%%%
\begin{frame}
  \begin{figure}
    \centering
    \subfloat[Angulo de iluminación solar\label{fig:a}]{\includegraphics[width=4cm]{respuesta1}}\qquad
    \subfloat[Pendiente y orientación de las laderas\label{fig:b}]{\includegraphics[width=4cm]{respuesta2}}
    \label{fig:1}
  \end{figure}
  \begin{figure}
    \centering
    \subfloat[Condiciones ambientales\label{fig:c}]{\includegraphics[width=4cm]{G:/My Drive/CATEDRA/ANALISIS GEOESPACIAL/fig/respuesta3}}\qquad    
    \subfloat[Condiciones atmosféricas\label{fig:d}]{\includegraphics[width=4cm]{G:/My Drive/CATEDRA/ANALISIS GEOESPACIAL/fig/respuesta4}}
    \label{fig:2}
  \end{figure}
\end{frame}
%##############################SLIDE########################################### 
 \begin{frame}
\frametitle{Firma Espectral}
\justifying
\scriptsize{La firma espectral se define como el comportamiento diferencial que presenta la radiación reflejada (reflectancia) o emitida (emitancia) desde algún tipo de superficie u objeto terrestre en los distintos rangos del espectro electromagnético. Una forma gráfica de estudiar este comportamiento es disponer los datos de reflectancia (\%) en el eje Y y la longitud de onda $\lambda$ en el eje X. Al unir los puntos con una línea continua se origina una representación bidimensional de la firma espectral}
 \begin{figure}
    \centering
    \includegraphics[height=.6\textheight]{G:/My Drive/CATEDRA/ANALISIS GEOESPACIAL/fig/firma}
  \end{figure}
\end{frame}
%%%%%%%%%%%%%%%%%%%%%%%%%%%%%%%%%%%%%%%%%%%%%%%%%%%%%%%%%%%%%%
\begin{frame}
%\framesubtitle{}
  \begin{figure}
    \centering
    \includegraphics[height=.8\textheight]{G:/My Drive/CATEDRA/ANALISIS GEOESPACIAL/fig/firma2}
  \end{figure}
\end{frame}
%################################SLIDE
\begin{frame}
%\framesubtitle{}
  \begin{figure}
    \centering
    \includegraphics[height=.8\textheight]{firma3}
  \end{figure}
\end{frame}
%################################SLIDE
\begin{frame}
\frametitle{Vegetación}
  \begin{figure}
    \centering
    \includegraphics[height=.8\textheight]{firma4}
  \end{figure}
\end{frame}
%################################SLIDE
\begin{frame}
\frametitle{Vegetación}
  \begin{figure}
    \centering
    \includegraphics[scale=0.5]{vegetacion1}
  \end{figure}
\end{frame}
%################################SLIDE
\begin{frame}
\frametitle{Vegetación}
  \begin{figure}
    \centering
    \includegraphics[scale=0.48]{vegetacion2}
  \end{figure}
\end{frame}
%################################SLIDE
\begin{frame}
\frametitle{Agua \& Sedimentos}
  \begin{figure}
    \centering
    \includegraphics[height=.8\textheight]{G:/My Drive/CATEDRA/ANALISIS GEOESPACIAL/fig/firma5}
  \end{figure}
\end{frame}
%################################SLIDE
\begin{frame}
\frametitle{Suelo \& Humedad}
  \begin{figure}
    \centering
    \includegraphics[height=.8\textheight]{G:/My Drive/CATEDRA/ANALISIS GEOESPACIAL/fig/firma6}
  \end{figure}
\end{frame}
%################################SLIDE
\begin{frame}
\frametitle{Suelo \& Granulometría}
  \begin{figure}
    \centering
    \includegraphics[height=.8\textheight]{G:/My Drive/CATEDRA/ANALISIS GEOESPACIAL/fig/firma7}
  \end{figure}
\end{frame}
%################################SLIDE
\begin{frame}
\frametitle{Firma espectral Minerales}
  \begin{figure}
    \centering
    \includegraphics[height=.8\textheight]{firmaespectral_minerales}
  \end{figure}
\end{frame}
%################################SLIDE
\begin{frame}
\frametitle{Firma espectral Minerales}
  \begin{figure}
    \centering
    \includegraphics[height=.8\textheight]{firmaespectral_minerales1}
  \end{figure}
\end{frame}
%################################SLIDE
%################################SLIDE
\begin{frame}
\frametitle{Emisividad: rocas igneas en el infrarojo}
\framesubtitle{Viedman Volcano}
  \begin{figure}
    \centering
    \includegraphics[height=.7\textheight]{ejemplo_emision}
  \end{figure}
  \tiny{Fuente: Kobayashi et al (2010)}
\end{frame}
%################################SLIDE
\end{document}